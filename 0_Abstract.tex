\begin{abstract}

\noindent
Tumor hypoxia is characterized by an insufficient oxygen concentration in certain localized regions of a tumor. Hypoxic cancer cells develop higher therapy resistance and greater migratory capability, thereby decreasing the effectiveness of cancer treatment methods, such as radiotherapy. Positron Emission Tomography (PET) imaging has recently been a major focus of research for its use as an \textit{in vivo} hypoxia detection tool. Hypoxia PET provides an oxygen concentration map that is overlaid on a Computed Tomography (CT) scan to enhance contrast around hypoxic regions, thereby enabling their localization and can therefore be integrated into radiotherapy workflow to account for tumor hypoxia and adjust the radiation dosage accordingly. HX4-PET is an instance of hypoxia PET imaging that uses a newly developed radiotracer denoted as [$^{18}$F]HX4. Despite its great potential in improving cancer treatment, HX4-PET imaging is expensive and is currently only limited to clinical trials. 
This work investigates a GAN-based computational alternative to hypoxia imaging. We apply image translation GANs to synthesize whole HX4-PET images from the more readily available FDG-PET and CT modalities and explore both paired and unpaired translation approaches. Using the paired Pix2Pix as a reference method, unpaired translation with CycleGAN is studied and compared. We argue that naively applying the default CycleGAN system to our use case is a flawed strategy because the invertibility assumption of CycleGAN is seriously violated here, and propose a design modification to the CycleGAN system for circumventing this issue and optimizing it for our purpose. We perform an extensive evaluation of the three methods by first testing them on a simulated translation task, followed by comprehensively evaluating on an appropriate 3D medical image dataset. We, additionally, perform a set of clinically relevant downstream tasks on the synthetic HX4-PET images to determine their clinical value.
    
\end{abstract}