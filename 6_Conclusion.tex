\chapter{Conclusion}
\label{Conclusion}

In this work, we posed the problem of predicting hypoxia from FDG-PET and CT scans as image-to-image translation, and investigated image translation GANs for synthesizing full HX4-PET images from the multimodal input. Using the paired Pix2Pix and the unpaired CycleGAN, we observed that the paired approach produces superior quality images both in our simulated translation task and in the HX4-PET synthesis task. We argue that the naive application of CycleGAN to the HX4-PET synthesis task has a conceptual flaw related to non-invertibility, and propose an alternative strategy to circumvent this problem while still requiring only unpaired training data. This modified CycleGAN system showed substantial performance improvement over the default CycleGAN in terms of image quality. To assess the quality of the synthetic HX4-PET images in a comprehensive manner, we construct a set of six measures of image quality and image similarity, of which three metrics -- MSE, MAE and PSNR -- measure the voxel-level accuracy of the the synthetic images with respect to the ground truth, whereas the remaining three -- SSIM, NMI and histogram distance -- account for the fidelity of image structure and statistics. A systematic visual inspection of the synthetic HX4-PET images validated the assessment of these metrics and also revealed common failure modes for each model. The Pix2Pix-generated images contained, on average, the least amount of degradation and artifacts, although they suffered from structural breaks in specific areas. These faults can be attributed to the supervised training of the Pix2Pix model which used ground truth images containing registration imperfections. Despite the clear performance difference across the three GAN models based on image quality, the clinical evaluation of their synthetic images conducted via tumor hypoxia quantification tasks produced mixed and inconclusive results. The Pix2Pix model performed remarkably worse in comparison with the two CycleGANs. While a majority of the tumors in the synthetic images produced by the CycleGAN models were classified correctly, the classification rate in Pix2Pix predictions is lower. However, as indicated by a poor segmentation score, none of the models could predict accurately the spatial distribution of high hypoxia. The primary reason could be the lack of sufficient training data, and the second reason, specifically for Pix2Pix, could be the noise in its supervision signal caused due to imperfect ground truth images. Given sufficient training data, unpaired models like CycleGAN, especially the modified version of it, could be more suitable for the HX4-PET synthesis task.